\documentclass[a4paper]{article}
\usepackage[T1]{fontenc}
\usepackage[utf8]{inputenc}
\usepackage[italian]{babel}
\usepackage{float, mathtools, ebproof, graphicx, xcolor}

\title{%
	CuCh machine \\
  	\large Linguaggi di Programmazione \\ 
	a.a. 2019-2020}

\author{Edoardo De Matteis  \\ 1746561 
   \and Mirko Giacchini \\ matricola } 
   
\date{}

\begin{document}
	\maketitle
	\tableofcontents
	
	\section{Introduzione}
	Una \textbf{Cu}rry-\textbf{Ch}urch machine implementa un interprete minimale di un linguaggio funzionale non tipato.   
	\section{Sintassi}
	\textit{\textcolor{gray}{syntax.sml}} \\
	 
	\begin{equation} \label{fun}
	\begin{split}
		FUN ::= & \ Const \ k \ |\ Var \ x \\
			&|\ Sum(M, N) \ |\ Fn(x, M) \\ 
			&|\ Let(x, M, N) \ |\ App(M, N) 
	\end{split}
	\end{equation}

	\begin{equation}
		K ::= 0 \ | \ 1 \ | \ \cdots
	\end{equation}
	
	\begin{equation}
		X ::= A \ | \ \cdots \ | \ Z \ | \ a \ | \ \cdots \ | \ z \ | \ \cdots
	\end{equation}

	\begin{equation} \label{env}
		ENV : VAR \rightharpoonup VAR \times FUN \times ENV
	\end{equation}

	\begin{equation} \label{find}
		find: ENV \times FUN \rightarrow (FUN \times ENV) \cup EXC
	\end{equation}
	
	In $ENV$ nel codominio il prodotto cartesiano presenta  $ENV$ poichè necessario nelle valutazioni con scoping statico, nel mondo dinamico non è necessario e semplicemente lo si ignora. $EXC$ è l'insieme delle eccezioni.

	\section{Semantica operazionale}
	\begin{equation} \label{eval}
		\mapsto \ \subseteq ENV \times FUN \times VAL \ \equiv \ ENV \vdash FUN \mapsto VAL
	\end{equation}

	\subsection{Dynamic eager}
	\textit{\textcolor{gray}{dynamic\_eager.sml}} \\

	%costanti
	\[
		\begin{prooftree}
			\hypo{}
			\infer1[]{E \vdash Const \ k \mapsto Const \ k}
		\end{prooftree}
	\]
	\\
	%variabile
	\[
		\begin{prooftree}
			\hypo{}
			\infer1[\quad E(x) = v]{E \vdash Var \ x \mapsto v}
		\end{prooftree}
	\]
	\\
	%somma
	\[
		\begin{prooftree}
			\hypo{E \vdash M \mapsto v1}
			\hypo{E \vdash N \mapsto v2 }
			\infer2[\quad (v=v1+v2)]{E \vdash Sum(M, N) \mapsto v}
		\end{prooftree}
	\]
	\\
	%funzione
	\[
		\begin{prooftree}
			\hypo{}
			\infer1[]{E \vdash Fn(x, M) \mapsto (x, M)}
		\end{prooftree}
	\]
	\\
	%applicazione
	\[
		\begin{prooftree}
			\hypo{E \vdash M \mapsto (x, M')}
			\hypo{E \vdash N \mapsto v}
			\hypo{E(x, v) \vdash M' \mapsto v'}
			\infer3[]{E \vdash App(M, N) \mapsto v'}
		\end{prooftree}
	\]
	\\
	%let
	\[
		\begin{prooftree}
			\hypo{E \vdash M \mapsto v}
			\hypo{E(x, v) \vdash N \mapsto v'}
			\infer2[]{E \vdash Let(x, M, N) \mapsto v' }
		\end{prooftree}
	\]

	\subsection{Dynamic lazy}
	\textit{\textcolor{gray}{dynamic\_lazy.sml}} \\
	%costanti
	\[
		\begin{prooftree}
			\hypo{}
			\infer1[]{E \vdash Const \ k\mapsto Const \ k}
		\end{prooftree}
	\]
	\\
	%variabile
	\[
		\begin{prooftree}
			\hypo{E \vdash M \mapsto v }
			\infer1[\quad E(x) = M]{E \vdash Var &x \mapsto v}
		\end{prooftree}
	\]
	\\
	%somma
	\[
		\begin{prooftree}
			\hypo{E \vdash M \mapsto v1}
			\hypo{E \vdash N \mapsto v2 }
			\infer2[\quad (v=v1+v2)]{E \vdash Sum(M, N) \mapsto v}
		\end{prooftree}
	\]
	\\
	%funzione
	\[
		\begin{prooftree}
			\hypo{}
			\infer1[]{E \vdash Fn(x, M) \mapsto (x, M)}
		\end{prooftree}
	\]
	\\
	%applicazione
	\[
		\begin{prooftree}
			\hypo{E \vdash M \mapsto (x, M')}
			\hypo{E(x, N) \vdash M' \mapsto v}
			\infer2[]{E \vdash App(M, N) \mapsto v}
		\end{prooftree}
	\]
	\\
	%let
	\[
		\begin{prooftree}
			\hypo{E(x,M) \vdash N \mapsto v}
			\infer1[]{E \vdash Let(x, M, N) \mapsto v }
		\end{prooftree}
	\]
	

	\subsection{Static eager}
	\textit{\textcolor{gray}{static\_eager.sml}} \\
	%costanti
	\[
		\begin{prooftree}
			\hypo{}
			\infer1[]{E \vdash Const \ k \mapsto Const \ k}
		\end{prooftree}
	\]
	\\
	%variabile
	\[
		\begin{prooftree}
			\hypo{E \vdash M \mapsto v }
			\infer1[\quad E(x) = M]{E \vdash Var &x \mapsto v}
		\end{prooftree}
	\]
	\\
	%somma
	\[
		\begin{prooftree}
			\hypo{E \vdash M \mapsto v1}
			\hypo{E \vdash N \mapsto v2 }
			\infer2[\quad (v=v1+v2)]{E \vdash Sum(M, N) \mapsto v}
		\end{prooftree}
	\]
	\\
	%funzione
	\[
		\begin{prooftree}
			\hypo{}
			\infer1[]{E \vdash Fn(x, M) \mapsto (x, M, E)}
		\end{prooftree}
	\]
	\\
	%applicazione
	\[
		\begin{prooftree}
			\hypo{E \vdash M \mapsto (x, M', E')}
			\hypo{E \vdash N \mapsto v}
			\hypo{E'(x, v) \vdash M' \mapsto v'}
			\infer3[]{E \vdash App(M, N) \mapsto v'}
		\end{prooftree}
	\]
	\\
	%let
	\[
		\begin{prooftree}
			\hypo{E \vdash M \mapsto v}
			\hypo{E(x, v) \vdash N \mapsto v'}
			\infer2[]{E \vdash Let(x, M, N) \mapsto v' }
		\end{prooftree}
	\]

	\subsection{Static lazy}
	\textit{\textcolor{gray}{static\_lazy.sml}} \\
	%costanti
	\[
		\begin{prooftree}
			\hypo{}
			\infer1[]{E \vdash Const \ k \mapsto Const \ k} 
		\end{prooftree}
	\]
	\\
	%variabile
	\[
		\begin{prooftree}
			\hypo{E' \vdash M \mapsto v }
			\infer1[\quad E(x) = (M, E')]{E \vdash Var &x \mapsto v}
		\end{prooftree}
	\]
	\\
	%somma
	\[
		\begin{prooftree}
			\hypo{E \vdash M \mapsto v1}
			\hypo{E \vdash N \mapsto v2 }
			\infer2[\quad (v=v1+v2)]{E \vdash Sum(M, N) \mapsto v}
		\end{prooftree}
	\]
	\\
	%funzione
	\[
		\begin{prooftree}
			\hypo{}
			\infer1[]{E \vdash Fn(x, M) \mapsto (x, M, E)}
		\end{prooftree}
	\]
	\\
	%applicazione
	\[
		\begin{prooftree}
			\hypo{E \vdash M \mapsto (x, M', E')}
			\hypo{E'(x, N) \vdash M' \mapsto v}
			\infer2[]{E \vdash App(M, N) \mapsto v}
		\end{prooftree}
	\]
	\\
	%let
	\[
		\begin{prooftree}
			\hypo{E(x, M, E) \vdash N \mapsto v}
			\infer1[]{E \vdash Let(x, M, N) \mapsto v}
		\end{prooftree}
	\]
	
	\section{Osservazioni}
	Un programma che mette in mostra le differenze tra una valutazione eager ed una lazy è il programma 
	\[ let \ x = x \ in \ x \]
	che in una semantica static eager (come in SML) dà errore perchè $x$ non è definita. In una semantica dynamic lazy va invece in loop

	\begin{figure}[ht]
	\[
		\begin{prooftree}
			\hypo{\cdots}
			\infer1[]{\langle (x,x) \rangle \vdash x \mapsto }
			\infer1[]{\langle (x,x) \rangle \vdash x \mapsto }
			\infer1[]{\emptyset \vdash Let(x,x,x) \mapsto }
		\end{prooftree}
	\]
	\end{figure}

	Inoltre in SML non è possibile eseguire il più piccolo transinfinito 
	\[ \omega = (fn \ x \Rightarrow x \ x)(fn \ x \Rightarrow x \ x) \]
	per via del sistema dei tipi, $FUN$ non essendo tipato non presenta questo problema e si entra in loop come si può dimostrare ad esempio con valutazione eager statica
	
	\begin{figure}[ht]
	\[
		\begin{prooftree}
			\hypo{\emptyset \vdash Fn(x, x \ x) \mapsto (x, x \ x, \emptyset)}

			\hypo{\emptyset \vdash Fn(x, x \ x) \mapsto (x, x \ x, \emptyset)}

			\hypo{\cdots}
			\infer1[]{\langle (x, x \ x) \rangle \vdash x \ x \mapsto}
			\infer1[]{\langle (x, x \ x) \rangle \vdash x \mapsto}
			\hypo{}
			\infer2[]{\langle (x, x \ x) \rangle \vdash x \ x \mapsto }

			\infer3[]{\emptyset \vdash Fn(x, x \ x)\ Fn(x, x \ x) \mapsto }

		\end{prooftree}
	\]
	\end{figure}
	
	Un'esecuzione interessante è quella di $\langle (x, k) \rangle \vdash App(Fn(x,x),x)$ per la cattura di $x$. 

	\begin{figure}[H]
	\[
		\begin{prooftree}
			\hypo{\langle (x,3) \rangle \vdash Fn(x,x) \mapsto (x,x)}
			\hypo{\cdots}
			\infer1[]{\langle (x,3)(x,x) \rangle \vdash x \mapsto }
			\infer1[]{\langle (x,3)(x,x) \rangle \vdash x \mapsto }
			\infer2[]{\langle (x,3) \rangle \vdash App(Fn(x,x),x) \mapsto }
		\end{prooftree}
	\]
	\end{figure}


	\begin{figure}[H]
	\[
		\begin{prooftree}
			\hypo{\langle (x,3) \rangle \vdash Fn(y,y) \mapsto (y,y)}
			\hypo{\langle (x,3)(y,y) \rangle \vdash x \mapsto 3}
			\infer2[]{\langle (x,3) \rangle \vdash App(Fn(y,y),x) \mapsto 3}
		\end{prooftree}
	\]
	\end{figure}

\end{document}
